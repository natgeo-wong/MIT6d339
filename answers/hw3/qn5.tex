\textbf{Problem 5.}

We consider a linear finite element discretization of the Neumann problem: Find $u$ such that
\begin{alignat}{2}
    -u_{xx} &= f \qquad&& \text{in $\Omega=(0,1)$} \\
    u(0)    &= 0 \\
    u_x(1)  &= g
\end{alignat}

for a given $f,g$ on a triangulation $\mathcal{T}_h$ of equi-sized elements $T_h^k$, $k=1,\dots,K = n$; the corresponding $n+1$ nodes are given by $x_0=0,\dots, x_n = 1$

\begin{enumerate}[label=(\alph*),leftmargin=*,itemsep=0mm]
    
    \item We first define $X_h$ and the associated nodal basis.  To do that we find the weak form of the PDE:
    \begin{gather*}
        a(u,v) = l(v), \qquad \forall\>v\in X
    \end{gather*}
    
    Multiply (27) by $v$ and performing integration by parts, we get
    \begin{align*}
        \int_0^1 -vu_{xx} \dd{x} &= \int_0^1 fv \dd{x} \\
        &= [-vu_x]_0^1 + \int_0^1 v_xu_x \dd{x}
    \end{align*}
    
    So, by applying the boundary condition that $v(0)=0$, we see that
    \begin{align*}
        \int_0^1 v_xu_x \dd{x}
        &= [vu_x]_0^1 + \int_0^1 fv \dd{x} \\
        &= g\cdot v(1) + \int_0^1 fv \dd{x}
    \end{align*}
    
    Which therefore gives us the bilinear and linear functions:
    \begin{align}
        a(w,w) = \int_\Omega w_x^2\dd{x}, \qquad
        l(w) = \int_0^1 fw \dd{x} + g\cdot w(1)
    \end{align}
    
    Where $X = \{v\in H^1(\Omega) \mid v(0) = 0\}$.  And since $X \subset H^1(\Omega)$, only the 1st derivative integrals need to be bounded, and therefore
    \begin{align}
        X_h = \{ v\in X \mid v|_{T_h^k} \in \mathbb{P}^1(T_h^k), \> k = 1, \dots, n\}
    \end{align}
    
    We define a nodal basis, similar to the one in the notes, such that $\phi_i$, such that at $x_j$, where $i,j = 1, \dots, n$
    \begin{align}
        \phi_i(x_j) = \begin{cases}
        1, & \text{for $i=j$} \\
        0, & \text{for $i\neq j$}
        \end{cases}
    \end{align}
    
    \item We see that $\forall\> v \in X_h$
    \begin{align}
        v = \sum_i v_i\phi_i(x),\quad
        u_h = \sum_j u_{h,j} \phi_j(x)
    \end{align}
    
    And from the weak form, we have that
    \begin{align}
        a(u,v) = a\left(\sum_j u_{h,j} \phi_j(x),\sum_i v_i\phi_i(x)\right)
        = l \left(\sum_i v_i\phi_i(x)\right) = l(v)
    \end{align}
    
    So by the properties of bilinearity and linearity, we have that
    \begin{align}
        \sum_{j=1}^n \sum_{i=1}^n v_i \cdot a(\phi_i,\phi_j) \cdot u_{h,j}
        = \sum_{i=1}^n v_i \cdot l(\phi_i)
    \end{align}
    
    As defined earlier, we have that $v_i = (0,...,1,...,0)$, where 1 occurs at the $i$-th element, so we have that the $i$-th row in the system of linear equations is defined by
    \begin{gather}
        \sum_j^n A_{h|i,j} \cdot u_{h|j} = F_{h|j} \Rightarrow A_h u_h = F_h \\
        A_h \in \mathbb{R}^{n\times n}, \quad
        u_h \in \mathbb{R}^{n}, \quad
        F_h \in \mathbb{R}^{n}
    \end{gather}
    
    where $A_{h|i,j} = a(\phi_i,\phi_j)$ and $F_{h,i} = l(\phi_i)$.
    
    Given that
    \begin{align}
        a(\phi_i,\phi_j) &= \int_\Omega \partial_x\phi_i \cdot \partial_x\phi_j \dd{x} \nonumber \\
        &= \int_{T_h^i} \partial_x\phi_i \cdot \partial_x\phi_j \dd{x}
        + \int_{T_h^{i+1}} \partial_x\phi_i \cdot \partial_x\phi_j \dd{x} \nonumber \\
        &= \partial_x\phi_i|_{x_i} \cdot \partial_x\phi_j|_{x_i} \cdot (x_i-x_{i-1})
        + \partial_x\phi_i|_{x_{i+1}} \cdot \partial_x\phi_j|_{x_{i+1}} \cdot (x_{i+1}-x_i) \nonumber \\\
        &= h \cdot ( \partial_x\phi_i|_{x_i} \cdot \partial_x\phi_j|_{x_i}
        + \partial_x\phi_i|_{x_{i+1}} \cdot \partial_x\phi_j|_{x_{i+1}} ) \\
        l(\phi_i) &= \int_\Omega f\cdot\phi_i \dd{x} + g\cdot \phi_i(1) \nonumber \\
        &= \int_{T_h^i} f\cdot\phi_i \dd{x} + \int_{T_h^{i+1}} f\cdot\phi_i \dd{x} + g\cdot \phi_i(1) \nonumber \\
        &= \begin{cases}
        \int_{T_h^i} f\cdot\phi_i \dd{x} + \int_{T_h^{i+1}} f\cdot\phi_i \dd{x} & i \neq n\\
        \int_{T_h^i} f\cdot\phi_n \dd{x} + g\cdot \phi_i(1) & i = n
        \end{cases}
    \end{align}
    \newpage
    There are basically three cases:
    \begin{itemize}
        \item \textit{Case 1:} The interior points/nodes, which are basically the same as what was given in the notes, so we have that
        \begin{align*}
            A_{h|i,i} &= \left(\frac{1}{h}\right)^2 \cdot h
            + \left(-\frac{1}{h}\right)^2 \cdot h
            = \frac{2}{h}\\
            A_{h|i,i-1} &= \frac{1}{h} \cdot \left(-\frac{1}{h}\right) \cdot h
            + \left(-\frac{1}{h}\right) \cdot 0 \cdot h
            = -\frac{1}{h} \\
            A_{h|i,i+1} &= \frac{1}{h} \cdot 0 \cdot h
            + \left(-\frac{1}{h}\right) \cdot \frac{1}{h} \cdot h
            = -\frac{1}{h} \\
            F_{h,i} &= l(\phi_i) = \int_{T_h^i} f\cdot\phi_i \dd{x} + \int_{T_h^{i+1}} f\cdot\phi_i \dd{x}
        \end{align*}
        
        Therefore the final equation for interior rows become
        \begin{align}
            \frac{2u_{h|i}-u_{h|i+1}-u_{h|i-1}}{h}
            = \int_{T_h^i} f\cdot\phi_i \dd{x} + \int_{T_h^{i+1}} f\cdot\phi_i \dd{x}
        \end{align}
        
        \item \textit{Case 2:} The leftmost bound $i=1$, where we have
        \begin{align*}
            A_{h|1,1} &= \left(\frac{1}{h}\right)^2 \cdot h
            + \left(-\frac{1}{h}\right)^2 \cdot h
            = \frac{2}{h}\\
            A_{h|1,2} &= \frac{1}{h} \cdot 0 \cdot h
            + \left(-\frac{1}{h}\right) \cdot \frac{1}{h} \cdot h
            = -\frac{1}{h} \\
            F_{h,1} &= l(\phi_1) = \int_{T_h^1} f\cdot\phi_1 \dd{x} + \int_{T_h^2} f\cdot\phi_1 \dd{x}
        \end{align*}
        
        Therefore the final equation for first row becomes
        \begin{align}
            \frac{2u_{h|1}-u_{h|2}}{h}
            = \int_{T_h^1} f\cdot\phi_1 \dd{x} + \int_{T_h^2} f\cdot\phi_1 \dd{x}
        \end{align}
        
        \item \textit{Case 3:} The rightmost bound $i=n$, where we have that $\partial_x\phi_{n+1}$ does not exist such that
        \begin{align*}
            A_{h|n,n} &= \left(\frac{1}{h}\right)^2 \cdot h = \frac{1}{h}\\
            A_{h|n,n-1} &= \frac{1}{h} \cdot \left(-\frac{1}{h}\right) \cdot h = -\frac{1}{h}\\
            F_{h,n} &= \int_{T_h^n} f\cdot\phi_n \dd{x} + g\cdot \phi_n(1)
            = \int_{T_h^n} f\cdot\phi_n \dd{x} + g
        \end{align*}
        
        Therefore the final equation for final row becomes
        \begin{align}
            \frac{u_{h|n}-u_{h|n-1}}{h}
            = \int_{T_h^n} f\cdot\phi_n \dd{x} + g
        \end{align}
        
    \end{itemize}
    
    \item In finite difference, at the $n$-th equation, we supply a $n+1$ ghost node such that
    \begin{align*}
        \frac{u_{n+1} - u_{n-1}}{2h} = g
    \end{align*}
    
    Recognizing that the 2nd derivative is given by
    \begin{align*}
        \frac{2u_n-u_{n+1}-u_{n-1}}{h^2} = f_n
    \end{align*}
    
    We obtain by combining the two:
    \begin{align*}
        \frac{2u_n-(u_{n+1}-u_{n-1})-2u_{n-1}}{h^2} = \frac{2u_n-2gh-2u_{n-1}}{h^2} = f_n
    \end{align*}
    
    Therefore,
    \begin{align*}
        \frac{u_n-u_{n-1}}{h^2} = \frac{f_n}{2} + \frac{g}{h} \Rightarrow \frac{u_n-u_{n-1}}{h} = \frac{f_nh}{2} + g
    \end{align*}
    
\end{enumerate}